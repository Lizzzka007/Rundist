\chapter{Рундист.}

\begin{definition}[Рундист]
	{\color{Black} \textbf{Рундист}} -- набор граней $R_0 = \{ Face_i\}_{i \in T_0} \in \{ R_i\},$ удовлетворяющий наибольшему числу следующих свойств:

	\begin{enumerate}
		\item $R_0$ имеет наибольшее число элементов среди остальных $R_i$:
		$$\begin{gathered}
		R_0 := \underset{R \in \{R_i\}}{argmax}(|R|)
		\end{gathered}$$

		\item полоса $R_0$ имеет наменьший размах среди остальных $R_i$:
		$$\begin{gathered}
		R_0 := \underset{R \in \{R_i\}}{argmin}(Amplitude(R))
		\end{gathered}$$

		\item угол между $l^{(up)}$ и $l^{(low)}$, соответствующий $R_0$, наименьший среди остальных $R_i$:
		$$\begin{gathered}
		R_0 := \underset{R \in \{R_i\}}{argmin}(sin(R))
		\end{gathered}$$

		\item через ось, перпендикулярную $\Pi(R_0)$ и проходящую через центр многогранника P, проходит наибольшее количество плоскостей симметрии относительно других $\Pi(R_i)$: 
		$$\begin{gathered}
		R_0 := \underset{R \in \{R_i\}}{argmax}(Sym(R))
		\end{gathered}$$

		\item набор $R_0$ лучше всех остальных $R_i$ аппроксимируется соответствующим цилиндром:
		$$\begin{gathered}
		R_0 := \underset{R \in \{R_i\}}{argmax}(S_{Cyl}(R))
		\end{gathered}$$

	\end{enumerate}
\end{definition}

\begin{remark}
	Константы, которые должны быть заданы:
	\begin{itemize}
		\item $N_{cs}$ -- количество сечений плоскостями для вычисления меры симметричности;
		\item n -- количество отражений для одного сечения при вычислении меры симметричности;
	\end{itemize}
\end{remark}
